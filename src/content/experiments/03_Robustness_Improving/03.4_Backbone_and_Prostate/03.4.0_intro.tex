\subsection{Backbone-NCA on Prostate}
\label{experiments:03.4.0:backbone_prost:intro}
In \autoref{experiments:03.1.1:backbone_hippo:spike_noise}, it was shown that the DiceBceNQM loss can improve the robustness of the Backbone-NCA on the hippocampus dataset. In \autoref{experiments:03.2.0:med_prost:intro}, we saw unclear results regarding whether this could be transferred to a more complex model \textit{and} another dataset, so we decided to investigate this separately. In \autoref{experiments:03.3.0:med_hippo:intro_and_Augmented}, we already looked at the transferability of the results from the hippocampus dataset to another model, the Med-NCA. In this section, we want to look at the transferability to another dataset with the same model as in \autoref{experiments:03.1.1:backbone_hippo:spike_noise}. We, therefore, tested the behavior of the Backbone NCA on the prostate datasets. We tested this on both augmented data (\autoref{experiments:03.4.1:backbone_prost:Augmented}) and domain shifts (\autoref{experiments:03.4.2:Backbone_prost:DomainShifts}). 
The results of the augmentations and domain shifts were similar to those of the Med-NCA, suggesting that the prostate dataset, rather than the model, is more challenging in transferability than the hippocampus dataset, but this would need to be verified with additional datasets. In particular, the all joined dataset is (partially) worse here, which reduces the results with the Med-NCA in this respect.


%%% --- inputs ---
\subsubsection{Augmented Datasets}
\label{experiments:03.4.1:backbone_prost:Augmented}
As with the Med-NCA in the second trial, we trained a total of 3k epochs each for the comparison. The comparison model was trained directly on the DiceBCE for 3k epochs. The test model was pretrained for 2k epochs on the DiceBCE and then trained for another 1k epochs on the DiceBceNQM. The results can be seen in \autoref{tab:03.4.1:Backbone_Prost:on_Spike}. We have trained a total of one cohort on the prostate spikes.

The results are similar to the augmentations with the Med-NCA (3k epochs) \autoref{tab:03.2.1:medNCA_Prost:on_Spike:3kepochs}. However, the models diverge even more. Overall, however, the trend does not look worse. Only the model trained on the fully augmented dataset Spike 1.0 performs significantly worse on the DiceBceNQM, with -6 points on the Dice. On the original dataset, as well as on Spike 0.3 and Spike 0.1, it tended to perform better, with +3 and +4 points, respectively. On the other two datasets, especially the attractive, less augmented dataset Spike 0.01, the model performed 1-2 points worse. This was similar for the Med-NCA, although the scores were distributed slightly differently, suggesting that the dataset rather than the model is more challenging in terms of transferability than the hippocampus dataset, but this would need to be verified with more datasets.

%%% aggregated table ... no use here %%%
\iffalse
\begin{table}[h!]
    \centering
    \begin{tabular}{|l|l|l|l|l|}
        \hline
        \bfseries & mean on signed & mean on absolute & sum on absolute & range\\\hline
        % ---
        \bfseries & 0.0008 & 0.0136 & 0.49 & (-0.055, 0.04)\\\hline
        % ---
    \end{tabular}
    \caption{Agreggations over the improvements$(+)$ and deteriorations$(-)$, using the DiceBceNQM compared to the DiceBCE on Dice. For the Backbone-NCA on the Spike Postate Datasets. All values of the experiments used for this Aggregations: \autoref{tab:03.4.1:Backbone_Prost:on_Spike}}
    \label{tab:3.4.1:Backbone_Prost:on_Spike_aggregated}
\end{table}
\fi
%%% tables %%%
\iftable
\begin{table}[H]
    \centering
    \begin{tabular}{ll!{\vrule width 1.3pt}llllll}
        \toprule
        \multicolumn{2}{c!{\vrule width 1.3pt}}{model} &
        \multicolumn{5}{c}{\textbf{test dataset} (Dice $\uparrow$)}\\\midrule
        {\bfseries train loss} & \textbf{train set} & original & Spike 1.0 & Spike 0.3 & Spike 0.2 & Spike 0.1 & Spike 0.01\\\midrule[1.3pt]
        % ---
        DiceBCE        & original    & 0.788 & 0.277 & 0.661 & 0.717 & 0.788 & 0.787\\
        DiceBCENQM     & original    & 0.798 +.01 & 0.297 +.02 & 0.663 & 0.719 & 0.804 +.02 & 0.795 +.01\\\rowcolor{BG}
        DiceBCE        & Spike 1.0   & 0.781 & 0.656 & 0.678 & 0.788 & 0.780 & 0.777\\\rowcolor{BG}
        DiceBceNQM     & Spike 1.0   & 0.783 & 0.601 \textbf{-.06} & 0.699 +.02 & 0.757 \textbf{-.03} & 0.769 -.01 & 0.774\\
        DiceBCE        & Spike 0.3   & 0.746 & 0.295 & 0.638 & 0.706 & 0.774 & 0.768\\
        DiceBceNQM     & Spike 0.3   & 0.776 \textbf{+.03} & 0.301 +.01 & 0.654 +.02 & 0.709 & 0.771 & 0.768\\\rowcolor{BG}
        DiceBCE        & Spike 0.2   & 0.798 & 0.301 & 0.677 & 0.725 & 0.804 & 0.798\\\rowcolor{BG}
        DiceBceNQM     & Spike 0.2   & 0.785 -.01 & 0.289 -.01 & 0.672 -.01 & 0.720 -.01 & 0.785 -.02 & 0.785 -.01\\
        DiceBCE        & Spike 0.1   & 0.795 & 0.354 & 0.686 & 0.751 & 0.797 & 0.801\\
        DiceBceNQM     & Spike 0.1   & 0.815 +.02 & 0.363 +.01 & 0.697 +.01 & 0.791 \textbf{+.04} & 0.819 +.02 & 0.822 +.02\\\rowcolor{BG}
        DiceBCE        & Spike 0.01  & 0.809 & 0.326 & 0.682 & 0.731 & 0.806 & 0.811\\\rowcolor{BG}
        DiceBceNQM     & Spike 0.01  & 0.796 -.01 & 0.321 -.01 & 0.669 -.01 & 0.724 -.01 & 0.798 -.01 & 0.796 -.02\\\bottomrule
    \end{tabular}
    \caption{Backbone-NCA, prostate dataset \textbf{Augmented with Spikes} (\autoref{experiments:03.4.1:backbone_prost:Augmented}): On Pretrained Model. Mixed results, as with the Med-NCA. Even more divergent, but overall not worse.}
    \label{tab:03.4.1:Backbone_Prost:on_Spike}
\end{table}
\begin{table}[H]
    \centering
    \begin{tabular}{ll!{\vrule width 1.3pt}llllll}
        \toprule
        \multicolumn{2}{c!{\vrule width 1.3pt}}{model} &
        \multicolumn{5}{c}{\textbf{test dataset} (Dice $\uparrow$)}\\\midrule
        {\bfseries train loss} & \textbf{train set}& original & Spike 1.0 & Spike 0.3 & Spike 0.2 & Spike 0.1 & Spike 0.01\\\midrule[1.3pt]
        % ---
        DiceBce       & original    & 0.794      & 0.368      & 0.683      & 0.712      & 0.801 & 0.795      \\
        DiceBceNQM    & original    & 0.800 +.01 & 0.367      & 0.693 +.01 & 0.727 +.02 & 0.802 & 0.808 +.01 \\\rowcolor{BG}
        DiceBce       & Spike 1.0   & 0.792      & 0.639      & 0.719      & 0.713      & 0.793 & 0.789      \\\rowcolor{BG}
        DiceBceNQM    & Spike 1.0   & 0.792      & 0.636      & 0.726 +.01 & 0.723 +.01 & 0.791 & 0.796 +.01 \\
        DiceBce       & Spike 0.3   & 0.791      & 0.584      & 0.674      & 0.710      & 0.793 & 0.793      \\
        DiceBceNQM    & Spike 0.3   & 0.783 -.01 & 0.637 \textbf{+.05} & 0.670      & 0.706      & 0.789 & 0.786 -.01 \\\rowcolor{BG}
        DiceBce       & Spike 0.2   & 0.782      & 0.399      & 0.668      & 0.711      & 0.789 & 0.792      \\\rowcolor{BG}
        DiceBceNQM    & Spike 0.2   & 0.786      & 0.421 +.02 & 0.669      & 0.713      & 0.790 & 0.789      \\
        DiceBce       & Spike 0.1   & 0.799      & 0.417      & 0.677      & 0.730      & 0.803 & 0.802      \\
        DiceBceNQM    & Spike 0.1   & 0.796      & 0.399 -.02 & 0.675      & 0.707 -.02 & 0.802 & 0.796 -.01 \\\rowcolor{BG}
        DiceBce       & Spike 0.01  & 0.795      & 0.382      & 0.677      & 0.720      & 0.797 & 0.793      \\\rowcolor{BG}
        DiceBceNQM    & Spike 0.01  & 0.797      & 0.369 -.01 & 0.678      & 0.711 -.01 & 0.793 & 0.795      \\\bottomrule
    \end{tabular}
    \caption{Med-NCA, prostate dataset \textbf{Augmented with Spikes} 1000 epochs on \textbf{pretrained model} (\autoref{experiments:03.2.1:med_prost:augmented}): A lot more stable, with this number of epochs. Except for one case, the DiceBceNQM no longer makes a difference here. Therefor the approach is stable now and no longer makes things worse in this setting.}
    \label{tab:03.2.1:medNCA_Prost:on_Spike:3kepochs}
\end{table}
\fi
\subsubsection{Domain Shifts}
\label{experiments:03.4.2:Backbone_prost:DomainShifts}
For the domain shifts, as in \autoref{experiments:03.2.2:med_prost:onDomainShifts}, we have one cohort each on the shift 1 datasets, as well as on the datasets where several additional samples were added, and on the all joined dataset (\autoref{experiments:03.0:Intro}). 
For the shift 1 datasets (\autoref{tab:03.4.2:Backbone_Prost:domainShifts:OneOnOriginals}), the results on the Backbon-NCA diverge much less than on the Med-NCA. The model trained on the original dataset tends towards more robustness on all datasets or at least no deterioration with DiceBceNQM. For the other models, it makes no difference whether DiceBCE or DiceBceNQM was used for training (max $\pm$ 3). 

Only the model trained on the all joined dataset shows an evident deterioration on one dataset (i2cvb) with -6 points, while the others are almost constant (max $\pm$ 2).

The results for the multiple domain shifts are also significantly less divergent ("only" $\pm$ 9, instead of $\pm$ 16) than for the Med-NCA but still similarly mixed, which again suggests that the prostate dataset is overall heavier than the hippocampus dataset, and clearly shows that the DiceBceNQM loss \textit{can} lead to a more robust model but also to a less robust model overall. And, that this seems to depend on the dataset, which is contrary to the idea of overall robustness.
\iffalse
%%% Singles %%%
XXX Singles ??? XXX 
-------------------
XXXX
--- Singels on Originals
mean: -0.0008
mean on absoulte values: 0.0164
sum: -0.019
sum on absoulte values: 0.411
range: (0.062, 0.059)


--- Singels on Singels
mean: -0.0027
mean on absoulte values: 0.0108
sum: -0.054
sum on absoulte values: 0.216
range: (-0.017, 0.024)

%%% Multis %%%
-------------------
XXXX
--- Multi on Originals
mean: 0.0038
mean on absoulte values: 0.0281
sum: 0.076
sum on absoulte values: 0.562
range: (-0.07, 0.087)

--- Multi on Multi
mean: 0.0081
mean on absoulte values: 0.0193
sum: 0.13
sum on absoulte values: 0.308
range: (-0.027, 0.066)
\fi
\iftable
%%%% tables %%%%
\begin{table}[H]
    \centering
    \begin{tabular}{ll!{\vrule width 1.3pt}llll}
        \toprule
        \multicolumn{2}{c!{\vrule width 1.3pt}}{model} &
        \multicolumn{4}{c}{\textbf{test dataset} (Dice $\uparrow$)}\\\midrule
        {\bfseries train loss} & \textbf{train dataset} & decath & decath-shift-1-isbi & decath-shift-1-i2cvb & decath-shift-1-ucl\\\midrule[1.3pt]
        % ---
        diceBce     & decath         & 0.781 & 0.805 & 0.813 & 0.811\\
        diceBceNQM  & decath         & 0.805 +.02 & 0.826 +.02 & 0.828 +.02 & 0.822 +.01\\\bottomrule %\rowcolor{BG}
        % diceBce     & shift 1 isbi   & 0.775 & 0.773 & 0.773 & 0.771\\\rowcolor{BG}
        % diceBceNQM  & shift 1 isbi   & 0.774 & 0.781 +.01 & 0.774 & 0.772\\
        % diceBce     & shift 1 i2cvb  & 0.769 & 0.781 & 0.767 & 0.774\\
        % diceBceNQM  & shift 1 i2cvb  & 0.758 -.01 & 0.767 -.01 & 0.759 -.01 & 0.757 -.02\\\rowcolor{BG}
        % diceBce     & shift 1 ucl    & 0.802 & 0.787 & 0.791 & 0.792\\\rowcolor{BG}
        % diceBceNQM  & shift 1 ucl    & 0.785 -.02 & 0.781 -.01 & 0.782 -.01 & 0.785 -.01\\
        % DiceBce     & all joined     & 0.919 & 0.922 & 0.923 & 0.922\\
        % DiceBceNQM  & all joined     & 0.905 -.01 & 0.913 -.01 & 0.912 -.01 & 0.911 -.01\\\hline
        % XXX Nur die Ersten Beiden Zeilen macht Sinn. Die anderen bietet keinen informationstechnischen Mehrwert, weil einfach andere nicht Domainfremde samples in test-set geschoben wurden
    \end{tabular}
    \caption{\textbf{Single Domainshifts} Test on Trainsets (\autoref{experiments:03.4.2:Backbone_prost:DomainShifts}): \todo{caption}}
    \label{tab:03.4.2:Backbone_Prost:domainShifts:OneOnOne}
\end{table}
\begin{table}[H]
    \centering
    \begin{tabular}{ll!{\vrule width 1.3pt}lllll}
        \toprule
        \multicolumn{2}{c!{\vrule width 1.3pt}}{model} &
        \multicolumn{5}{c}{\textbf{test dataset} (Dice $\uparrow$)}\\\midrule
        {\bfseries train loss} & \textbf{train dataset} & decath & isbi & i2cvb & ulc & all joined\\\midrule[1.3pt]
        % ---
        diceBce     & decath           & 0.776 & 0.427 & 0.236 & 0.313 & 0.370\\
        diceBceNQM  & decath           & 0.795 +.02 & 0.486 \textbf{+.06} & 0.254 +.02 & 0.322 +.01 & 0.383 +.01\\\rowcolor{BG}
        diceBce     & shift 1 isbi     & 0.782 & 0.541 & 0.252 & 0.327 & 0.424\\\rowcolor{BG}
        diceBceNQM  & shift 1 isbi     & 0.774 -.01 & 0.527 -.01 & 0.283 \textbf{+.03} & 0.327 & 0.402 -.02\\
        diceBce     & shift 1 i2cvb    & 0.767 & 0.354 & 0.274 & 0.332 & 0.355\\
        diceBceNQM  & shift 1 i2cvb    & 0.765 & 0.343 -.01 & 0.256 -.02 & 0.330 & 0.361 +.01\\\rowcolor{BG}
        diceBce     & shift 1 ucl      & 0.806 & 0.511 & 0.237 & 0.345 & 0.377\\\rowcolor{BG}
        diceBceNQM  & shift 1 ucl      & 0.786 -.02 & 0.483 \textbf{-.03} & 0.257 +.02 & 0.343 & 0.382 +.01\\
        DiceBce     & all joined       & 0.919 & 0.905 & 0.759 & 0.923 & 0.794\\
        DiceBceNQM  & all joined       & 0.904 -.02 & 0.894 -.01 & 0.697 \textbf{-.06} & 0.933 +.01 & 0.800 +.01\\\bottomrule
    \end{tabular}
    \caption{Backbone-NCA, \textbf{Single Domainshifts} and \textbf{all joined} (\autoref{experiments:03.4.2:Backbone_prost:DomainShifts}): Tests on original datasets. Very divergent and mixed results. But a lot less divergent then for the Med-NCA.}
    \label{tab:03.4.2:Backbone_Prost:domainShifts:OneOnOriginals}
\end{table}
\begin{table}[H]
    \centering
    \begin{tabular}{ll!{\vrule width 1.3pt}lllllll}
        \toprule
        \multicolumn{2}{c!{\vrule width 1.3pt}}{model} &
        \multicolumn{5}{c}{\textbf{test dataset} (Dice $\uparrow$)}\\\midrule
        {\bfseries train loss} & \textbf{train dataset} & decath & isbi & i2cvb & ulc & all joined\\\midrule[1.3pt]
        % ---
        diceBce     & decath          & 0.792 & 0.410 & 0.236 & 0.314 & 0.365\\
        diceBceNQM  & decath          & 0.789 & 0.497 \textbf{+.09} & 0.254 +.02 & 0.322 +.01 & 0.383 +.02\\\rowcolor{BG}
        diceBce     & shift 12 isbi   & 0.864 & 0.731 & 0.302 & 0.322 & 0.548\\\rowcolor{BG}
        diceBceNQM  & shift 12 isbi   & 0.877 +.01 & 0.714 -.02 & 0.259 \textbf{-.04} & 0.327 +.01 & 0.572 +.02\\
        diceBce     & shift 8 i2cvb   & 0.873 & 0.498 & 0.482 & 0.327 & 0.499\\
        diceBceNQM  & shift 8 i2cvb   & 0.867 -.01 & 0.428 \textbf{-.07} & 0.551 \textbf{+.07} & 0.333 +.01 & 0.494 -.01\\\rowcolor{BG}
        diceBce     & shift 4 ucl     & 0.750 & 0.659 & 0.250 & 0.681 & 0.582\\\rowcolor{BG}
        diceBceNQM  & shift 4 ucl     & 0.805 \textbf{+.06} & 0.619 \textbf{-.04} & 0.266 +.02 & 0.638 \textbf{-.04} & 0.566 -.02\\\bottomrule
    \end{tabular}
    \caption{Backbone-NCA, \textbf{Multiple Domainshifts} (\autoref{experiments:03.4.2:Backbone_prost:DomainShifts}): Tests on original datasets. As divergent and mixed results as Single Domainshifts. Less divergent as for the Med-NCA.}
    \label{tab:03.4.2:backbone_Prost:domainShifts:multiOnOriginals}
\end{table}
\begin{table}[H]
    \centering
    \begin{tabular}{ll!{\vrule width 1.3pt}llll}
        \toprule
        \multicolumn{2}{c!{\vrule width 1.3pt}}{model} &
        \multicolumn{4}{c}{\textbf{test dataset} (Dice $\uparrow$)}\\\midrule
        {\bfseries train loss} & \textbf{train dataset} & decath & decath shift 12 isbi  & decath shift 8 i2cvb & decath shift 4 ulc\\\midrule[1.3pt]
        % --
        diceBce     & decath            & 0.784 & 0.631 & 0.598 & 0.812\\
        diceBceNQM  & decath            & 0.796 +.01 & 0.655 +.02 & 0.602 & 0.825 +.01\\\bottomrule %\rowcolor{BG}
        % diceBce     & shift 12 isbi     & 0.863 & 0.786 & 0.601 & 0.879\\\rowcolor{BG}
        % diceBceNQM  & shift 12 isbi     & 0.881 +.02 & 0.759 -.03 & 0.622 +.02 & 0.885 +.01\\
        % diceBce     & shift 8 i2cvb     & 0.866 & 0.666 & 0.737 & 0.882\\
        % diceBceNQM  & shift 8 i2cvb     & 0.869 & 0.649 -.02 & 0.712 -.03 & 0.887 +.01\\\rowcolor{BG}
        % diceBce     & shift 4 ucl       & 0.738 & 0.708 & 0.578 & 0.748\\\rowcolor{BG}
        % diceBceNQM  & shift 4 ucl       & 0.804 +.07 & 0.688 -.02 & 0.594 +.02 & 0.779 +.03\\\hline
        % XXX Nur die Ersten Beiden Zeilen macht Sinn. Die anderen bietet keinen informationstechnischen Mehrwert, weil einfach andere nicht Domainfremde samples in test-set geschoben wurden
    \end{tabular}
    \caption{\textbf{Multiple Domainshifts} Test on Trainsets (\autoref{experiments:03.4.2:Backbone_prost:DomainShifts}): \todo{caption}}
    \label{tab:03.4.2:backbone_Prost:domainShifts:multiOnMulti}
\end{table}
\fi