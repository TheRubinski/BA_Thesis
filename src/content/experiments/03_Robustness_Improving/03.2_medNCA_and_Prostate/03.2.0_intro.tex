\subsection{Med-NCA on Prostate}
\label{experiments:03.2.0:med_prost:intro}
After \autoref{experiments:03.1.1:backbone_hippo:spike_noise} showed that the robustness of the backbone model on the hippocampus dataset can be improved with the DiceBceNQM, we tested whether this could be transferred to a more complex model and a different dataset. We used the Med-NCA (\autoref{methods:NCA:Med-NCA}) and the medical segmentation decathlon prostate dataset \cite{Antonelli:2022:MedSegmentationDecatlon}.\\
For this purpose, we artificially augmented the prostate dataset, as well as the hippocampus dataset before, with spikes and random noise, as described in \autoref{experiments:03.0:Intro}. We also generated a series of perturbed datasets with domain shifts for the prostate dataset. Again, as described in \autoref{experiments:03.0:Intro}. \\
However, this led to much more mixed results than the backbone NCA on the hippocampus dataset (\autoref{experiments:03.1.1:backbone_hippo:spike_noise}), as we will see in this subsection.
In all cohorts, there were some training datasets where DiceBceNQM provided significant improvements over DiceBCE alone, but also some datasets where there was significant degradation. 
In the augmentations, which initially showed enormous scatter on the fist training setup, the differences between DiceBCE and DiceBceNQM became significantly smaller by extending the training cycle. Not only, but also to the disadvantage of DiceBceNQM. 
The results of the domain shifts can be divided into two parts. On the one hand, there were similar or even worse results on datasets where only a few samples were added, but which we could not resolve or improve.
On the other hand, there were significant improvements with the DiceBceNQM on the dataset merged from several sources, the all joined.

However, not only because of the poor adaptation to the augmented and shifted datasets, the question arises whether much simpler techniques than using the DiceBceNQM instead of the DiceBCE might achieve more significant effects, such as extending the training time (in the best case up to convergence), increasing the batch size, or even targeted or selective batch duplication. Especially regarding the optimal use of available resources, the DiceBceNQM requires a lot of graphics memory due to the multiple output generation.


%%% --- inputs ---
\subsubsection{Augmented Datasets}
\label{experiments:03.2.1:med_prost:augmented}
For the augmentations on the prostate dataset, as with the hippocampus dataset, we augmented one series with noise and one series with spikes, as described in \autoref{experiments:03.0:Intro}. We trained a cohort of models on each of these using the standard setup form \cite{kalkhof:2023:medNCA}.

As can be seen in \ref{tab:03.2.1:medNCA_Prost:on_Noise} and \ref{tab:03.2.1:medNCA_Prost:on_Spike}, the results from \ref{experiments:03.1.1:backbone_hippo:spike_noise} do not generalize here. Overall, there are models in both cohorts that perform significantly better than DiceBCE (e.g., on Noise 0.2 and Spike 0.2), as well as models that perform significantly worse or at least very mixed (e.g., on Spike 1.0 and Noise 0.01). In particular, the models trained on very low-disturbance data (Spike 0.01 and Noise 0.01) perform worse with DiceBceNQM than with DiceBCE. 
Since the med-NCAs, here in the standard setup of \cite{kalkhof:2023:medNCA} with 1000 epochs, do not seem to have been trained to convergence, we trained the models of interest for the robustness question on the Spike 0.01 and Noise 0.01 datasets (last two lines in \autoref{tab:03.2.1:medNCA_Prost:on_Spike} and \autoref{tab:03.2.1:medNCA_Prost:on_Noise}). The results of the noise series have remained the same overall but shifted across the different datasets. For the spike dataset, they have improved slightly but were already slightly worse than the DiceBCE.

For the models trained on noise, the results are also more ambiguous. For the models trained on spikes, the results are more evident. Interestingly, for the models trained on spike, the models trained on Spike 0.2 and Spike 0.3 performed better with DiceBceNQM than those trained on DiceBCE alone. Overall, the models trained on the original dataset also performed worse on the noisy datasets than on the Backbone-NCA test series. It could be that these datasets are simply more challenging.

After seeing in \autoref{experiments:03.1.2:backbone_hippo:pretrained} that pretraining can bring an improvement, we tried it again with Med-NCA and prostate. But this time, only on the spike dataset. The results can be seen in \autoref{tab:03.2.1:medNCA_Prost:on_Spike:3kepochs}. Compared to the cohort trained with 1000 epochs, the scatter between the DiceBCE and DiceBceNQM models has decreased massively. The range of deviations is from $(-0.07, +0.14)$ to $(-0.02, +0.05)$. The sum of deviations (unsigned, absolute values) across the cohort decreased from 1.062 to 0.272.
At the same time, the DiceBceNQM shows an improvement of 5 points from 0.58 to 0.64 on one model (spike 0.3) but is otherwise neutral with max $\pm$ 2 points, especially on the interesting original and spike 0.01 datasets. 

In this way, however, we were able to eliminate the negative effects of the previous setting on the Med-NCA. For the Med-NCA, using 1000 epochs on a pretrained model is a lot more stable. Except for one case, the DiceBceNQM no longer makes a difference here ($\pm$2). Therefor the approach is stable now and no longer makes things worse in this setting.
So we are confident that the positive results from \autoref{experiments:03.1.1:backbone_hippo:spike_noise} with the DiceBceNQM regarding radiological noise can be generalized.
\iffalse
--- Spike 1k
mean: 0.0076
mean on absoulte values: 0.0295
sum: 0.272
sum on absoulte values: 1.062
range: (-0.069, +0.136)

--- Noise 1k
mean: 0.0002
mean on absoulte values: 0.0235
sum: 0.007
sum on absoulte values: 0.845
range: (-0.052, 0.081)

--- Spike 3k
mean: 0.0012
mean on absoulte values: 0.0076
sum: 0.042
sum on absoulte values: 0.272
range: (-0.023, 0.053)
\fi
%%% tables %%%
\begin{table}[H]
    \centering
    \begin{tabular}{ll!{\vrule width 1.3pt}llllll}
        \toprule
        \multicolumn{2}{c!{\vrule width 1.3pt}}{model} &
        \multicolumn{5}{c}{\textbf{test dataset} (Dice $\uparrow$)}\\\midrule
        {\bfseries train loss} & \textbf{train set}& original & Spike 1.0 & Spike 0.3 & Spike 0.2 & Spike 0.1 & Spike 0.01\\\midrule[1.3pt]
        % ---
        DiceBCE        & original    & 0.787 & 0.321 & 0.684 & 0.717 & 0.790 & 0.793\\
        DiceBceNQM     & original    & 0.772 -.02 & 0.384 \textbf{+.06} & 0.665 -.02 & 0.712 -.01 & 0.776 -.01 & 0.774 -.02\\\rowcolor{BG}
        DiceBCE        & Spike 1.0   & 0.776 & 0.593 & 0.722 & 0.678 & 0.767 & 0.780\\\rowcolor{BG}
        DiceBceNQM     & Spike 1.0   & 0.740 -.04 & 0.632 +.04 & 0.653 \textbf{-.07} & 0.666 -.01 & 0.740 -.03 & 0.740 -.04\\
        DiceBCE        & Spike 0.3   & 0.791 & 0.607 & 0.683 & 0.726 & 0.798 & 0.793\\
        DiceBceNQM     & Spike 0.3   & 0.780 -.01 & 0.658 \textbf{+.05} & 0.667 -.02 & 0.737 +.01 & 0.780 -.02 & 0.781 -.01\\\rowcolor{BG}
        DiceBCE        & Spike 0.2   & 0.774 & 0.377 & 0.656 & 0.701 & 0.776 & 0.771\\\rowcolor{BG}
        DiceBceNQM     & Spike 0.2   & 0.808 +.03 & 0.503 \textbf{+.13} & 0.693 +.04 & 0.709 +.01 & 0.810 +.03 & 0.809 +.04\\
        DiceBCE        & Spike 0.1   & 0.788 & 0.392 & 0.671 & 0.712 & 0.789 & 0.794\\
        DiceBceNQM     & Spike 0.1   & 0.802 +.01 & 0.528 \textbf{+.14} & 0.684 +.01 & 0.759 +.05 & 0.800 +.01 & 0.798\\\rowcolor{BG}
        DiceBCE        & Spike 0.01  & 0.807 & 0.353 & 0.686 & 0.721 & 0.802 & 0.807\\\rowcolor{BG}
        DiceBceNQM     & Spike 0.01  & 0.790 -.02 & 0.354 & 0.674 -.01 & 0.700 -.02 & 0.789 -.01 & 0.788 -.02\\\hline
        % --- 2k ---
        2k DiceBCE    & spike 0.01  & 0.793 & 0.345 & 0.684 & 0.711 & 0.797 & 0.790\\
        2k DiceBceNQM & spike 0.01  & 0.791 & 0.347 & 0.674 -.01 & 0.700 -.01 & 0.789 -.01 & 0.791\\\bottomrule
    \end{tabular}
    \caption{Med-NCA, prostate dataset \textbf{Augmented with Spikes} 1000 epochs (\autoref{experiments:03.2.1:med_prost:augmented}): Mixed and divergent result. With this setting the approach becomes unstabel. However, this could be changed with a pretrained model (\autoref{tab:03.2.1:medNCA_Prost:on_Spike:3kepochs}).}
    \label{tab:03.2.1:medNCA_Prost:on_Spike}
\end{table}
\begin{table}[H]
    \centering
    \begin{tabular}{ll!{\vrule width 1.3pt}llllll}
        \toprule
        \multicolumn{2}{c!{\vrule width 1.3pt}}{model} &
        \multicolumn{5}{c}{\textbf{test dataset} (Dice $\uparrow$)}\\\midrule
        {\bfseries train loss} & \textbf{train set}& original & Spike 1.0 & Spike 0.3 & Spike 0.2 & Spike 0.1 & Spike 0.01\\\midrule[1.3pt]
        % ---
        DiceBce       & original    & 0.794      & 0.368      & 0.683      & 0.712      & 0.801 & 0.795      \\
        DiceBceNQM    & original    & 0.800 +.01 & 0.367      & 0.693 +.01 & 0.727 +.02 & 0.802 & 0.808 +.01 \\\rowcolor{BG}
        DiceBce       & Spike 1.0   & 0.792      & 0.639      & 0.719      & 0.713      & 0.793 & 0.789      \\\rowcolor{BG}
        DiceBceNQM    & Spike 1.0   & 0.792      & 0.636      & 0.726 +.01 & 0.723 +.01 & 0.791 & 0.796 +.01 \\
        DiceBce       & Spike 0.3   & 0.791      & 0.584      & 0.674      & 0.710      & 0.793 & 0.793      \\
        DiceBceNQM    & Spike 0.3   & 0.783 -.01 & 0.637 \textbf{+.05} & 0.670      & 0.706      & 0.789 & 0.786 -.01 \\\rowcolor{BG}
        DiceBce       & Spike 0.2   & 0.782      & 0.399      & 0.668      & 0.711      & 0.789 & 0.792      \\\rowcolor{BG}
        DiceBceNQM    & Spike 0.2   & 0.786      & 0.421 +.02 & 0.669      & 0.713      & 0.790 & 0.789      \\
        DiceBce       & Spike 0.1   & 0.799      & 0.417      & 0.677      & 0.730      & 0.803 & 0.802      \\
        DiceBceNQM    & Spike 0.1   & 0.796      & 0.399 -.02 & 0.675      & 0.707 -.02 & 0.802 & 0.796 -.01 \\\rowcolor{BG}
        DiceBce       & Spike 0.01  & 0.795      & 0.382      & 0.677      & 0.720      & 0.797 & 0.793      \\\rowcolor{BG}
        DiceBceNQM    & Spike 0.01  & 0.797      & 0.369 -.01 & 0.678      & 0.711 -.01 & 0.793 & 0.795      \\\bottomrule
    \end{tabular}
    \caption{Med-NCA, prostate dataset \textbf{Augmented with Spikes} 1000 epochs on \textbf{pretrained model} (\autoref{experiments:03.2.1:med_prost:augmented}): A lot more stable, with this number of epochs. Except for one case, the DiceBceNQM no longer makes a difference here. Therefor the approach is stable now and no longer makes things worse in this setting.}
    \label{tab:03.2.1:medNCA_Prost:on_Spike:3kepochs}
\end{table}
%\begin{table}[H]
    \centering
    \begin{tabular}{ll!{\vrule width 1.3pt}llllll}
        \toprule
        \multicolumn{2}{c!{\vrule width 1.3pt}}{model} &
        \multicolumn{5}{c}{\textbf{test dataset} (Dice $\uparrow$)}\\\midrule
        {\bfseries train loss} & \textbf{train set} & original & Noise 1.0 & Noise 0.3 & Noise 0.2 & Noise 0.1 & Noise 0.01\\\midrule[1.3pt]
        % ---
        DiceBCE        & original    & 0.794 & 0.519 & 0.688 & 0.790 & 0.726 & 0.794\\
        DiceBceNQM     & original    & 0.774 -.02 & 0.559 +.04 & 0.683 & 0.774 -.02 & 0.726 & 0.767 -.03\\\rowcolor{BG}
        DiceBCE        & Noise 1.0   & 0.773 & 0.777 & 0.778 & 0.757 & 0.764 & 0.767\\\rowcolor{BG}
        DiceBceNQM     & Noise 1.0   & 0.754 -.02 & 0.780 & 0.774 & 0.754 & 0.754 -.01 & 0.755 -.01\\
        DiceBCE        & Noise 0.3   & 0.826 & 0.717 & 0.782 & 0.819 & 0.821 & 0.819\\
        DiceBceNQM     & Noise 0.3   & 0.819 -.01 & 0.746 +.03 & 0.802 +.02 & 0.819 & 0.820 & 0.821\\\rowcolor{BG}
        DiceBCE        & Noise 0.2   & 0.777 & 0.616 & 0.706 & 0.772 & 0.749 & 0.778\\\rowcolor{BG}
        DiceBceNQM     & Noise 0.2   & 0.798 +.02 & 0.678 \textbf{+.06} & 0.787 \textbf{+.08} & 0.797 +.03 & 0.798 \textbf{+.05} & 0.798 +.02\\
        DiceBCE        & Noise 0.1   & 0.798 & 0.757 & 0.780 & 0.794 & 0.779 & 0.795\\
        DiceBceNQM     & Noise 0.1   & 0.817 +.02 & 0.708 \textbf{-.05} & 0.778 & 0.818 +.02 & 0.790 +.01 & 0.815 +.02\\\rowcolor{BG}
        DiceBCE        & Noise 0.01  & 0.838 & 0.754 & 0.811 & 0.836 & 0.821 & 0.833\\\rowcolor{BG}
        DiceBceNQM     & Noise 0.01  & 0.800 -.04 & 0.720 -.03 & 0.763 \textbf{-.05} & 0.799 -.04 & 0.769 \textbf{-.05} & 0.798 -.03\\\hline
        % --- 2k --- 
        2k DiceBCE     & Noise 0.01  & 0.826 & 0.766 & 0.801 & 0.829 & 0.821 & 0.832\\
        2k DiceBceNQM  & Noise 0.01  & 0.798 -.03 & 0.694 \textbf{-.07} & 0.771 -.03 & 0.796 -.03 & 0.776 -.04 & 0.796 -.04\\\bottomrule
    \end{tabular}
    \caption{Med-NCA, prostate dataset \textbf{Augmented with Noise} 1000 epochs (\autoref{experiments:03.2.1:med_prost:augmented}): Mixed and divergent result. With this setting the approach becomes unstabel. However, this could be changed with a pretrained model for spikes (\autoref{tab:03.2.1:medNCA_Prost:on_Spike:3kepochs}). For noise this experiment was not transfered to the pretrained setting in the context of this thesis.}
    \label{tab:03.2.1:medNCA_Prost:on_Noise}
\end{table}
\subsubsection{Domain Shifts}
\label{experiments:03.2.2:med_prost:onDomainShifts}
For the domain shifts, we created two series of "shift" datasets, with the prostate dataset from the Medical Segmentation Decathlon \cite{Antonelli:2022:MedSegmentationDecatlon} as the base, in which individual or a few data samples were added, and an "all joined" dataset, in which we joined all data samples together. As described in \autoref{experiments:03.0:Intro}.


First, we trained a cohort on the shift 1 series and the all joined. The cohort's performance on the different original datasets can be seen in \autoref{tab:03.2.2:medNCA_Prost:domainShifts:OneOnOriginals}. On the decathlon, which is the primary dataset here, the performance is almost unchanged (max $\pm$2), which is ok. On the shift 1 datasets, however, the models trained on the DiceBceNQM show significantly worse results (up to -14 points on the dice). In some cases even on the datasets from which the samples were taken (worst case: shift 1 isbi with -11 points). In other words, the exact opposite effect occurred. The robustness decreased significantly in this experiment, even if the values were already in the lower mid-range with less than 50, sometimes less than 40 points. Surprisingly, the opposite is the case on the all connected. On the DiceBceNQM, the model performs significantly better than on the DiceBCE. Up to +7 points (ucl) on the individual datasets, although the model already had relatively high scores on the DiceBCE (82 points with ucl). Thus, in this experiment, the DiceBceNQM made a significant contribution to robustness.
The models trained only on the decathlon dataset without the shifts for comparison performed equally well with DiceBCE and DiceBceNQM on all test datasets. Overall, the


The comparison model trained with DiceBCE on the decathlon dataset barely shows a performance drop on the shift 1 datasets. Therefor, and because of the modest results here, we decided to investigate whether this changes on datasets with more samples shifted in. To do this, we generated a second set of shift datasets as described in \autoref{experiments:03.0:Intro} and trained a cohort on them. The results can be seen in \autoref{tab:03.2.2:medNCA_Prost:domainShifts:multiOnOriginals}. One model (shift 12 isbi) showed a positive effect. The model on the DiceBceNQM is no longer worse than on the DiceBCE. For the other two models, the models with the DiceBceNQM are no worse than with the DiceBCE, at least on the datasets from which the samples were shifted.


Overall, 1. the DiceBceNQM did not improve robustness when there were only a single non-domain samples in a dataset. However, 2. for mixed datasets from many domains, the DiceBceNQM led to a significant improvement in robustness (all joined).
\iffalse
--------------------------
XXX
--- Multi on Multi

mean: 0.002
mean on absoulte values: 0.0204
sum: 0.032
sum on absoulte values: 0.326
range: (-0.079, 0.038)

--- Multi on Originals
mean: -0.0154
mean on absoulte values: 0.0335
sum: -0.308
sum on absoulte values: 0.67
range: (-0.164, 0.057)
\fi
%%%% tables %%%%
\begin{table}[H]
    \centering
    \begin{tabular}{ll!{\vrule width 1.3pt}lllllll}
        \toprule
        \multicolumn{2}{c!{\vrule width 1.3pt}}{model} &
        \multicolumn{5}{c}{\textbf{test dataset} (Dice $\uparrow$)}\\\midrule
        {\bfseries train loss} & \textbf{train dataset} & decath & isbi & i2cvb & ucl & all joined\\\midrule[1.3pt]
        % ---
        diceBce     & decath          & 0.793 & 0.324 & 0.243 & 0.323 & 0.301\\
        diceBceNQM  & decath          & 0.794 & 0.329 +.01 & 0.253 +.01 & 0.325 & 0.287 -.01\\\rowcolor{BG}
        diceBce     & shift 1 isbi    & 0.772 & 0.441 & 0.256 & 0.321 & 0.367\\\rowcolor{BG}
        diceBceNQM  & shift 1 isbi    & 0.767 -.01 & 0.335 \textbf{-.11} & 0.249 -.01 & 0.324 & 0.306 \textbf{-.06}\\
        diceBce     & shift 1 i2cvb   & 0.736 & 0.340 & 0.343 & 0.327 & 0.356\\
        diceBceNQM  & shift 1 i2cvb   & 0.753 +.02 & 0.196 \textbf{-.14} & 0.376 +.03 & 0.334 +.01 & 0.290 -.07\\\rowcolor{BG}
        diceBce     & shift 1 ucl     & 0.804 & 0.407 & 0.247 & 0.326 & 0.320\\\rowcolor{BG}
        diceBceNQM  & shift 1 ucl     & 0.783 -.02 & 0.321 \textbf{-.09} & 0.258 +.01 & 0.331 +.01 & 0.271 \textbf{-.05}\\
        DiceBce     & all joined      & 0.904 & 0.812 & 0.658 & 0.821 & 0.722\\
        DiceBceNQM  & all joined      & 0.903 & 0.824 +.01 & 0.684 +.03 & 0.888 \textbf{+.07} & 0.728 +.01\\\bottomrule
    \end{tabular}
    \caption{Med-NCA, \textbf{Single Domainshifts} and \textbf{all joined} (\autoref{experiments:03.2.2:med_prost:onDomainShifts}): Test on original datasets. For the all joined dataset the DiceBceNQM is significantly more robust. It is also more instable on the datasets with single volume shifts.}
    \label{tab:03.2.2:medNCA_Prost:domainShifts:OneOnOriginals}
\end{table}
\begin{table}[H]
    \centering
    \begin{tabular}{ll!{\vrule width 1.3pt}lllllll}
        \toprule
        \multicolumn{2}{c!{\vrule width 1.3pt}}{model} &
        \multicolumn{5}{c}{\textbf{test dataset} (Dice $\uparrow$)}\\\midrule
        {\bfseries train loss} & \textbf{train dataset} & decath & isbi & i2cvb & ucl & all joined\\\midrule[1.3pt]
        % ---
        diceBce     & decath         & 0.789 & 0.306 & 0.246 & 0.323 & 0.308\\
        diceBceNQM  & decath         & 0.788 & 0.333 +.03 & 0.253 +.01 & 0.325 & 0.290 -.02\\\rowcolor{BG}
        diceBce     & shift 12 isbi  & 0.838 & 0.645 & 0.331 & 0.318 & 0.516\\\rowcolor{BG}
        diceBceNQM  & shift 12 isbi  & 0.849 +.01 & 0.643 & 0.388 \textbf{+.06} & 0.319 & 0.514\\
        diceBce     & shift 8 i2cvb  & 0.852 & 0.438 & 0.578 & 0.328 & 0.459\\
        diceBceNQM  & shift 8 i2cvb  & 0.859 +.01 & 0.304 \textbf{-.13} & 0.536 -.04 & 0.323 -.01 & 0.411 -.05\\\rowcolor{BG}
        diceBce     & shift 4 ucl    & 0.781 & 0.561 & 0.280 & 0.519 & 0.505\\\rowcolor{BG}
        diceBceNQM  & shift 4 ucl    & 0.812 +.03 & 0.397 \textbf{-.16} & 0.263 -.02 & 0.557 +.04 & 0.449 \textbf{-.06}\\\bottomrule
    \end{tabular}
    \caption{Med-NCA, \textbf{Multiple Domainshifts} (\autoref{experiments:03.2.2:med_prost:onDomainShifts}): Test on Original Datasets. Even more mixed and divergent as Single Domainshifts.}
    \label{tab:03.2.2:medNCA_Prost:domainShifts:multiOnOriginals}
\end{table}